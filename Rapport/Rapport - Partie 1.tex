\documentclass{article}
\usepackage[utf8]{inputenc}
\usepackage[T1]{fontenc}
\usepackage{url}
\usepackage[margin=3cm]{geometry}
\usepackage{verbatim}
\usepackage[francais]{babel}
\usepackage{graphicx}
\newcommand\tab[1][1cm]{\hspace*{#1}}

\title{INFO-H303 - Projet - Partie 1}
\author{Wets Loukas, Alfaro Perez Victor, Muller Noëmie}


\begin{document}
\maketitle

\renewcommand{\abstractname}{}


\section{Modèle entité-association}


\begin{figure}[hb]
\begin{center}
\includegraphics[scale=0.4]{image.png} 
\end{center}
\caption{Modèle entité-association}
\end{figure}
\
\subsection*{Contraintes d'intégrité}

- La \textit{position initiale} de la \textit{trottinette} doit être à Bruxelles.

- La \textit{position finale} de la \textit{trottinette} doit être un numéro unique.

- L'\textit{indentifiant} de chaque \textit{trottinette}, de chaque \textit{utilisateur} et de chaque \textit{technicien} doit etre un numéro unique.

- Les \textit{trottinettes} ne doivent pas être tout à fait chargées pour être rechargées.

- La \textit{recharge} doit être éffectué entre 22h et 7h du matin.

- L'\textit{inspection} des trottinettes doit être effectué entre 22h et 7h du matin.

- Une \textit{trottinette} ne peut être utilisé que par un \textit{utilisateur} à la fois.

- Une \textit{trottinette} ne peut être réparé que par un \textit{technicien} à la fois.

- Une \textit{trottinette} ne peut être rechargé que par un \textit{utilisateur qui recharge} à la fois.

- Une \textit{trottinette} ne peut pas être \textit{utilisé} et être en même temps en train de \textit{chargé} ou en train d'être \textit{réparé}.

- La \textit{Date de réparation} doit être supérieur à la \textit{date de dépot de plainte}.

- La \textit{Date de mise en service} doit être inférieur à la \textit{date de dépot de plainte}.

- La \textit{Date d'intervention} doir être supérieur ou égale a la \textit{date de dépot de plainte}.


\section{Modèle Relationnel}
Trottinette(\underline{N\textsuperscript{o}Trottinette}, DateMiseService, N\textsuperscript{o}Modèle, Etat, Identifiant)

\tab Trottinette.DateMiseEnService <= Intervention.DateDeDépot

\bigbreak
Utilisateur(\underline{Identifiant, MotDePasse}, N\textsuperscript{o}Carte)
\bigbreak
Technicien(\underline{N\textsuperscript{o}Employé}, Nom, Prenom, N\textsuperscript{o}Téléphone, DateEmbauche
\bigbreak
Trajet (\underline{HeureDeVerrouillage, N\textsuperscript{o}Trottinette, Identifiant}, HeurVerouillage, PositionInitiale, PositionFinale)

\tab N\textsuperscript{o}Trottinette référence Trottinette.N\textsuperscript{o}Trottinette

\tab Identifiant référence Utilisateur.Identifiant
\bigbreak
UtilisateurQuiRecharge(\underline{Identifiant}, Nom, n\textsuperscript{o}Téléphone, AdresseRue, AdresseNumero, AdresseCommune, AdresseCodePostale)

\tab Identifiant référence Utilisateur.idnetifiant
\bigbreak
Recharge(\underline{Identifiant, N\textsuperscript{o}Trottinette, Date}, ChargeInitiale, ChargeFinale, Position, Heure)

\tab Identifiant référence UtilisateurQuiRecharge.Identifiant

\tab N\textsuperscript{o}Trottinette référence Trottinette.n\textsuperscript{o}Trottinette
 \bigbreak
 Intervention(\underline{N\textsuperscript{o}Trotinette, N\textsuperscript{o}Employe, DateReparation}, DateDéppotPlainte, Utilisateur, Note, RetirerTrottnette)
 
 \tab N\textsuperscript{o}Trottinette référence Trottinette.N\textsuperscript{o}Trottinette
 
\tab N\textsuperscript{o}Employé référence Technicien.N\textsuperscript{o}Trottinette

\tab Intervention.DateRéparation >= Intervention.DateD2potPlainte
\bigbreak
TrottinetteUtilisateur(\underline{N\textsuperscript{o}Trottinette, Identifiant})

\tab N\textsuperscript{o}Trottinette référence Trottinette.N\textsuperscript{o}Trottinette

\tab Identifiant référence Utilisateur.Identifiant
\bigbreak
UtilisateurTechnicien(\underline{Identifiant, N\textsuperscript{o}Employé})

\tab Identifiant référence Utilisateru.Identifient

\tab N\textsuperscript{o}Employe référence Technicien.N\textsuperscript{o}Employé


\end{document}
difftime|Ops\.factor|Ops\.numeric_version|Ops\.ordered|Ops\.POSIXt|options|order|ordered|outer|packageEvent|packageHasNamespace|packageStartupMessage|package_version|packBits|pairlist|parent\.env|parent\.frame|parse|parseNamespaceFile|paste|paste0|path\.expand|path\.package|pi|pipe|pmatch|pmax|pmax\.int|pmin|pmin\.int|polyroot|Position|pos\.to\.env|pretty|prettyNum|print|print\.AsIs|print\.by|print\.condition|print\.connection|print\.data\.frame|print\.Date|print\.difftime|print\.DLLInfo|print\.DLLInfoList|print\.DLLRegisteredRoutines|print\.factor|print\.function|print\.hexmode|print\.libraryIQR|print\.listof|print\.NativeRoutineList|print\.noquote|print\.numeric_version|print\.octmode|print\.packageInfo|print\.POSIXct|print\.POSIXlt|print\.proc_time|print\.restart|print\.rle|print\.simple\.list|print\.srcfile|print\.srcref|print\.summaryDefault|print\.summary\.table|print\.table|print\.warnings|prmatrix|proc\.time|prod|prop\.table|psigamma|pushBack|pushBackLength|q|qr|qr\.coef|qr\.fitted|qr\.Q|qr\.qty|qr\.qy|qr\.R|qr\.resid|qr\.solve|qr\.X|quarters|quarters\.Date|quarters\.POSIXt|quit|quote|range|rank|rapply|raw|rawConnection|rawConnectionValue|rawShift|rawToBits|rawToChar|rbind|rbind\.data\.frame|rcond|Re|readBin|readChar|read\.dcf|readline|readLines|readRDS|readRenviron|real|Recall|Reduce|regexec|regexpr|reg\.finalizer|registerS3method|registerS3methods|regmatches|remove|removeCConverter|removeTaskCallback|rep|rep\.Date|rep\.factor|rep\.int|replace|replicate|rep\.numeric_version|rep\.POSIXct|rep\.POSIXlt|require|requireNamespace|restartDescription|restartFormals|retracemem|rev|R\.home|rle|rm|RNGkind|RNGversion|round|round\.Date|round\.POSIXt|row|rowMeans|rownames|row\.names|row\.names\.data\.frame|rowsum|rowsum\.data\.frame|rowSums|R_system_version|R\.version|R\.Vera